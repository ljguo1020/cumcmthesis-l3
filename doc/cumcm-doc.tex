\documentclass{cumcmdoc}
\usepackage{xeCJKfntef, xpinyin}
\usepackage{graphicx}
\usepackage{array,tabularx}
\usepackage{tabularray}
\title{\textcolor{MaterialIndigo800}{%
  \textbf{cumcmthesis: \href{http://www.mcm.edu.cn}{全国大学生数学建模竞赛} \LaTeX \xpinyin[font=\sffamily,format=\color{MaterialIndigo800}]{模}{mu2}板}}}
\author{郭李军}
\date{2022/07/15\quad v0.1.1%
  \thanks{%
    \url{https://github.com/ljguo1020/cumcmthesis-l3}
  }
}
\begin{document}
\maketitle
\vskip 1cm%
\begin{center}
  \includegraphics[width = 0.45\textwidth]{cumcm-logo.png}
\end{center}
\newpage
\tableofcontents
\section{介绍}
本模板为\textbf{全国大学生数学建模竞赛}\LaTeX{}论文(非官方)模板,旨在帮助建模选手能够摆脱 word 的烦恼,更加轻松的编写论文.

\textcolor{red}{\textbf{QQ交流群:762431803}}

目前网上可以找到的较老的模板:
\begin{enumerate}
  \item LaTeXstudio \href{https://github.com/latexstudio/CUMCMThesis}{CUMCMThesis} 
  \item sikouhjw \href{https://github.com/sikouhjw/JXUSTmodeling}{JXUSTmodeling}
\end{enumerate}
本模板借鉴了前面两位前辈的不少劳动成果,并在其基础之上优化了一下接口,使代码看起来更加美观,暂时没有什么新功能。
\subsection{\LaTeX{}入门}
网上的入门教程多如牛毛,这里我只推荐 \href{https://www.ctan.org/pkg/lshort-zh-cn}{lshort-zh-cn} 这本手册,另外安装问题可以阅读 \href{https://www.ctan.org/pkg/install-latex-guide-zh-cn}{install-latex-guide-zh-cn}。
\subsection{安装及使用}
请使用\cls{texlive}最新版本,然后将 \cls{cumcmthesis.cls} 放在您的工作文件夹内,然后在新建文件 \cls{mian.tex} 写下如下内容:
\begin{latexcode}[deletetexcs={\documentclass},
  moretexcs={\chapter},morekeywords={\documentclass},
  emph={[2]document}]
  % main.tex
  \documentclass{cumcmthesis}
  \begin{document}
    \maketitle
    hello \LaTeX{}!
  \end{document}
\end{latexcode}
然后在命令行使用
\begin{shellcode}[morekeywords={xelatex}]
  xelatex main.tex
\end{shellcode}
即可得到一份简单的编译结果。

\section{接口}
\begin{function}[added = 2022-07-15]{\cumcmsetup}
  \begin{ccnusyntax}[emph={[1]cumcmsetup}]
    \cumcmsetup(*\marg{键值列表}*)
  \end{ccnusyntax}
该命令用在导言区,用于设置模板的一些参数,具体如下
\end{function}
\begin{function}[added = 2022-07-15]{titlepage}
  \begin{ccnusyntax}[emph={[1]titlepage}]
    titlepage = (*\TTF*)
  \end{ccnusyntax}
  用于控制是否显示承诺书和编号页,默认是显示
\end{function}

\begin{function}[added = 2022-07-15]{info}
  \begin{ccnusyntax}[emph={[1]info}]
    info = (*\marg{键值列表}*)
    info/(*\meta{key}*) = (*\meta{value}*)
  \end{ccnusyntax}
  用于设置信息
\end{function}

\begin{function}[added = 2022-07-15]{info/tihao}
  \begin{ccnusyntax}[emph={[1]tihao}]
    tihao =  (*\marg{题号}*)
  \end{ccnusyntax}
  用于设置题号
\end{function}

\begin{function}[added = 2022-07-15]{info/baominghao}
  \begin{ccnusyntax}[emph={[1]baominghao}]
    baominghao =  (*\marg{报名号}*)
  \end{ccnusyntax}
  用于设置报名号
\end{function}

\begin{function}[added = 2022-07-15]{info/school}
  \begin{ccnusyntax}[emph={[1]school}]
    school =  (*\marg{学校名称}*)
  \end{ccnusyntax}
  用于设置学校名称
\end{function}

\begin{function}[added = 2022-07-15]{info/member}
  \begin{ccnusyntax}[emph={[1]member}]
    member =  (*\marg{memberA,memberB,memberC}*)
  \end{ccnusyntax}
  用于设置3个队员,应该用西文逗号分隔
\end{function}

\begin{function}[added = 2022-07-15]{info/supervisor}
  \begin{ccnusyntax}[emph={[1]supervisor}]
    supervisor =  (*\marg{指导老师}*)
  \end{ccnusyntax}
  用于设置指导老师
\end{function}

\begin{function}[added = 2022-07-15]{info/date}
  \begin{ccnusyntax}[emph={[1]date}]
    date =  (*\marg{年,月,日}*)
  \end{ccnusyntax}
  用于设置日期,默认为当前日期
\end{function}

\begin{function}[added = 2022-07-15]{info/title}
  \begin{ccnusyntax}[emph={[1]title}]
    title =  (*\marg{论文题目}*)
  \end{ccnusyntax}
  用于设置论文题目,默认为\textbf{全国大学生数学建模竞赛 \LaTeX{} 论文模板}
\end{function}

\begin{function}[added = 2022-07-15]{abstract}
  \begin{ccnusyntax}[emph={[1]abstract}]
    \begin{abstract}
      ...
    \end{abstract}
  \end{ccnusyntax}
  摘要环境
\end{function}

\begin{function}[added = 2022-07-15]{\keywords}
  \begin{ccnusyntax}[emph={[1]keywords}]
    \keywords(*\marg{关键词}*)
  \end{ccnusyntax}
  关键词
\end{function}

\begin{function}[added = 2022-07-15]{python}
  \begin{ccnusyntax}[emph={[1]python}]
    \begin{python}
      ...
    \end{python}
  \end{ccnusyntax}
  python代码抄录环境
\end{function}

\begin{function}[added = 2022-07-15]{matlab}
  \begin{ccnusyntax}[emph={[1]matlab}]
    \begin{matlab}
      ...
    \end{matlab}
  \end{ccnusyntax}
  matlab代码抄录环境
\end{function}

\section{使用样例}
\begin{latexcode}[deletetexcs={\documentclass},
  moretexcs={\chapter},morekeywords={\documentclass,\cumcmsetup},
  emph={[2]document}]
  \documentclass{cumcmthesis}
  \usepackage{zhlipsum}
  \cumcmsetup
  {
    info={
     tihao = {A},
     baominghao = {1123201},
     school = {中国科学技术大学},
     member = {张三,李四,王二麻子},
     supervisor = {王有明}, 
    }
  }
  \begin{document}
  
  \maketitle
  
  \begin{abstract}
    \zhlipsum[1]
  \end{abstract}
  \keywords{关键词1;关键词2;关键词3;关键词4}
  
  \newpage
  
  \section{模板}
  
  \begin{figure}[h]
    \centering
    \includegraphics[width=0.45\textwidth]{example-image-a}
    \caption{这是一张图片}\label{fig1}
  \end{figure}
  
  \begin{table}[h]
    \centering
    \caption{这是一个表格}\label{tab1}
    \begin{tabular}{ccccc}
      \toprule
      元素1-1 & 元素1-2 & 元素1-3\\
      \midrule
      元素2-1 & 元素2-2 & 元素2-3\\
      元素3-1 & 元素3-2 & 元素3-3\\
      \bottomrule
    \end{tabular}
  \end{table}

  \subsection{代码}
  \begin{matlab}
    x = 0:pi/10:2*pi;
    y1 = sin(x);
    y2 = sin(x-0.25);
    y3 = sin(x-0.5);
  
    figure
    plot(x,y1,'g',x,y2,'b--o',x,y3,'c*')
  \end{matlab}
  
  \end{document} 
\end{latexcode}

\section{已载入宏包}
本模板为用户默认载入了如表\ref{tab1}中所展示的宏包,无需重复加载
\begin{table}[h]
\centering
\caption{模板预载入宏包}\label{tab1}
\begin{tabularx}{\textwidth}%
{*{6}{>{\centering\arraybackslash}X}}
\toprule
geometry    &    graphicx   &   xeCJKfntef  &    calc       &    multicol   &   tikz        \\
listings    &    caption    &   subcaption  &    amsmath    &    amsfonts   &   amssymb     \\
float       &    booktabs   &   multirow    &    url        &    enumitem   &   etoolbox    \\
hyperref    &    amsthm     &   makecell    &               &               &               \\
\bottomrule
\end{tabularx}
\end{table}
\end{document}